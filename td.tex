1. Addition de deux matrices
Début
    Lire les dimensions des matrices (lignes, colonnes)
    Lire la matrice A
    Lire la matrice B
    Créer une matrice résultat C de mêmes dimensions

    Pour chaque ligne i de 0 à lignes-1:
        Pour chaque colonne j de 0 à colonnes-1:
            C[i][j] = A[i][j] + B[i][j]
    
    Afficher la matrice C
Fin

2. Soustraction de deux matrices
Début
    Lire les dimensions des matrices (lignes, colonnes)
    Lire la matrice A
    Lire la matrice B
    Créer une matrice résultat C de mêmes dimensions

    Pour chaque ligne i de 0 à lignes-1:
        Pour chaque colonne j de 0 à colonnes-1:
            C[i][j] = A[i][j] - B[i][j]
    
    Afficher la matrice C
Fin

3.Multiplication de deux matrices
Début
    Lire les dimensions de la matrice A (lignes_A, colonnes_A)
    Lire les dimensions de la matrice B (lignes_B, colonnes_B)
    
    Si colonnes_A != lignes_B:
        Afficher "Multiplication impossible"
    Sinon:
        Créer une matrice résultat C de dimensions (lignes_A, colonnes_B)

        Pour chaque ligne i de 0 à lignes_A-1:
            Pour chaque colonne j de 0 à colonnes_B-1:
                C[i][j] = 0
                Pour chaque k de 0 à colonnes_A-1:
                    C[i][j] += A[i][k] * B[k][j]
        
        Afficher la matrice C
Fin

4.Transposée d'une matrice

Algorithme:  Transposée d'une matrice
Variables: i , j : entier
Tableau: T[ , ] 


Début
    Lire les dimensions de la matrice A (lignes, colonnes)
    Créer une matrice résultat C de dimensions (colonnes, lignes)

    Pour chaque ligne i de 0 à lignes-1:
        Pour chaque colonne j de 0 à colonnes-1:
            C[j][i] = A[i][j]
    
    Afficher la matrice C
Fin